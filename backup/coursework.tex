\documentclass{scrartcl}

% \usepackage[sexy]{evan}
\usepackage{cole}
\setlength{\parskip}{0.5em}
% \setlength{\footskip}{130pt}

\title{List of Coursework}
\author{(Gautier) Cole Killian}


\begin{document}

\maketitle

\section*{McGill}

\subsection*{Computer Science}

\begin{itemize}
  \item[TBD] \textbf{Comp 417}, \textit{Intro to Robotics and Intelligent Systems},
    Fall 2020, David Meger

    Geometric motion planning, feedback control, perception, mapping, and state estimation.

    \textit{\small Textbook: Introduction to Robotics, Mechanics and Control, 3rd edition, by Craig.}

  \item[TBD] \textbf{Comp 321}, \textit{Programming Challenges},
    Fall 2020, David Becerra

    Test algorithm design and programming skills on tricky problems and puzzles.

    \textit{\small Textbook: Programming Challenges}

  \item[TBD] \textbf{Comp 310}, \textit{Operating Systems},
    Fall 2020, Muthucumaru Maheswaran

    Theoretical and practical concepts behind modern operating systems. Processes, inter-process communication, scheduling, memory management, virtual memory, storage management, network management, and security.

    \textit{\small Textbook: Silberchatz and Galvin, Operating System Concepts, 10th Edition, Wiley, 2018}

  \item[TBD] \textbf{Comp 302}, \textit{Programming Languages and Paradigms},
    Fall 2020, Brigitte Pientka 

    Functional programming and types. Ocaml.

    \textit{\small Textbook: B. C. Pierce: "Types and Programming Languages". MIT Press, 2002.}

  \item[A] \textbf{Comp 598}, \textit{Automata and Computability},
    Summer 2020, Prakash Panangaden

    Languages, automata, Kleene theorem, minimization, monoids, linear temporal logic, reductions, PCP, valcomps, logic and unsolvability, arithmetic hierarchy.

    \textit{\small Textbook: Automata and Computability by Dexter Kozen}
  \item[A] \textbf{Comp 252}, \textit{Honors Algorithms and Data Structures},
    Winter 2020, Luc Devroye

    Introduce student to algorithmic analysis, fundamental data structures, and
    problem solving paradigms.

    \textit{\small Textbook: T.H. Cormen, C.E.Leiserson, R.L.Rivest, and C. Stein: Introduction to Algorithms (Third Edition)}

  \item[A] \textbf{Comp 273}, \textit{Intro to Computer Systems},
    Winter 2020, Kaleem Siddiqi

    Number representations, combinational logic, sequential logic, MIPS assembly
    language and CPU architecture, Memory, I/O, Finite State Machines

    \textit{\small Textbook: Computer organization and design: the
      hardware/software interface by David A. Patterson and John L. Hennese}
    
  \item[A] \textbf{Comp 322}, \textit{Introduction to C++}, Winter 2020, Chad
    Zammar

    Covers the essential features of C++. Focus on pointers, memory
    allocation, tepmlates, classes, operator overloading, namespaces,
    exceptions, and the STL.

    \textit{\small Textbook: The C++ Programming Language by Bjarne Stroustrup}

  \item[A] \textbf{Comp 250}, \textit{Intro to Computer Science}, Fall 2019, Michael Langer and Giulia Alberino
  
  Learn  basic  data  structures  for  lists  (arrays,  linked  lists,  stacks,  queues),  trees  (search trees, heaps), and graphs. Analyze algorithms in terms of the amount of computation they use. Implementations in Java.
  
  \textit{\small Textbook: None}

  \item[A] \textbf{Comp 206}, \textit{Intro to Software Systems}, Fall 2019, Joseph Vybihal
  
  This course focuses on System Application Development, which relates to the integration of differing software, programming languages and environments into a single application. It provides a comprehensive introduction to and overview of the C programming language and how to use it with the UNIX environment to build software. In this light the course also teaches programming in Bash, interfacing with the operating system and interfacing with networking
  
  \textit{\small Textbook: Software Systems ed 3; Vybihal \& Azar; Kendall/Hunt}
\end{itemize} 

\subsection*{Math and Stats}

\begin{itemize}
  \item[TBD] \textbf{Math 356}, \textit{Honors Probability}, Fall 2020, Linan Chen
  
    Probability, discrete and continuous random variables, probability distributions, variance, moments and generating functions, multiple random variables, independence, correlation, condition, law of large numbers.
    
    \textit{\small Textbook: An Introduction to Probability and Statisticsby Rohatgi and Saleh, Wiley, 2015}

  \item[TBD] \textbf{Math 350}, \textit{Honors Discrete Math}, Fall 2020, Sergey Norin
  
    Fundamental concepts in graph theory: trees, matchings, connectivity, graph coloring, planar graphs. 
    
    \textit{\small Textbook: Introduction to Graph Theoryby D. West.}

  \item[A-] \textbf{Math 255}, \textit{Honors Analysis 2}, Winter 2020, Pengfei
    Guan

    Point set topology in metric space. Sequences, convergence, and continuity
    in general metric space. Normed vector spaces. Riemann-stieltjes integral.
    Infinite series. Uniform convergence of functions, Arzela-Ascoli Compactness
    Theorem, Stone-Weierstrass theorem.
    
    \textit{\small Textbook: Introduction to Real Analysis, Bartle and Sherbert.
    Principles of Mathematical Induction, Rudin}

  \item[A] \textbf{Math 251}, \textit{Honors Algebra 2}, Winter 2020, Henri Darmon

    Focuses on linear algebra. Linear maps and matrix representations.
    Determinants. Canonical forms. Duality. Bilinear and quadratic forms. Real
    and complex inner product spaces. Diagonalization of self-adjoint operators.

    \textit{\small Textbook: Linear Algebra and Geometry, Kostrikin and Manin}

  \item[A] \textbf{Math 254}, \textit{Honors Analysis 1}, Fall 2019,  Axel Hundeme 
  
  Logic, sets, functions, and other preliminaries. The Real Numbers. Sequences. Elementary Point-Set Topology. Limits and Continuity. Differentiation

  
  \textit{\small Textbook: Introduction to Real Analysis by R. Bartle and D. Sherbert, 4th edition, Wiley}
  % Real Analysis and Foundations , by Steven G. Krantz, 4th edition, Chapman and Hall.
  % Principles of Mathematical Analysis by W. Rudin, 3rd edition, McGraw-Hill.
  \item[A] \textbf{Math 235}, \textit{Algebra 1}, Fall 2019, Daniel Wise 
  
  Sets, functions and relations. Groups, subgroups and cosets; group actions on sets. Methods of proof. Complex numbers. Divisibility theory for integers and modular arithmetic. Divisibility theory for polynomials. Rings, ideals and quotient rings. Fields and construction of fields from polynomial rings. 

  \textit{\small Textbook: Abstract Algebra: Theory and Applications (2017 edition) by Tom Judson.}

  \item[A] \textbf{Math 222}, \textit{Calculus 3}, Summer 2020, Jerome Fortier
  
    Taylor series, Taylor’s theorem in one and several variables. Review of vector geometry. Partialdifferentiation, directional derivative.  Extreme of functions of 2 or 3 variables.  Parametriccurves and arc length. Polar and spherical coordinates. Multiple integrals
  
  \textit{\small Textbook: Stewart, J.,Calculus: Multivariable Calculus, 8th Edition}
  

  \item[A] \textbf{Math 133}, \textit{Linear Algebra}, Fall 2019, Rosalie Bélanger-Rioux 
  
  Linear transformations, matrices and vectors, linearindependence, subspaces and bases. Determinants, eigenvalues and eigenvectors, diagonalization. Study of linear systems of equations, their solutions, and the underlying structure of these problems.
  
  \textit{\small Textbook: W. K. Nicholson, Linear Algebra with Applications, Open Edition. Lyryx Learning Inc. Base Textbook, Version 2019, Revision A.}
  
\end{itemize} 

\subsection*{Misc}

\begin{itemize}
\item[A] \textbf{Chem 120}, \textit{General Chemistry 2}, Winter 2020, Mitchell
  Huot, Paul Wiseman, Sam Lewis Sewall, Pallavi Sirjoosingh
  
  Chemistry 120 aims to provide you with an introduction to the quantitative aspects of fundamental chemical principles, such as gas laws, thermodynamics, kinetics, solubility, equilibrium, and acids/base.
  
  \textit{\small Textbook: The Molecular Nature of Matter and Change (2nd
    Canadian Edition) by Silberberg, Amateis, Lavieri, and Venkateswaran}
\end{itemize}


\section*{Harvard}

\subsection*{Math and Stats}

\begin{itemize}
  \item[A] \textbf{Math 21b}, \textit{Linear Algebra and Differential Equations}, Spring 2019, Janet Chen 
  
  Matrices and related topics such as linear transformations and linear spaces, determinants, eigenvalues, and eigenvectors. Applications include dynamical systems including nonlinear systems, data fitting, ordinary and partial differential equations, and Fourier series.
  
  \textit{\small Textbook: Multivariable Calculus: Concepts and Contexts by James Stewart}
  
  \item[A] \textbf{Math 21a}, \textit{Multivariable Calculus}, Fall 2018, Janet Chen 
  
  Extending single variable calculus to higher dimensions. Develops methods for solving optimization problems with and without constraints.
  
  \textit{\small Textbook: Linear Algebra with Applications by Otto Bretscher}
  
\end{itemize} 

\subsection*{Physics}

\begin{itemize}
  \item[A] \textbf{Physics 101}, \textit{Foundations of Theoretical Physics }, Spring 2019, Jacob Barandes 

  A comprehensive, fast-paced introduction to the conceptual and mathematical foundations of modern theoretical physics that starts from the very beginning of the subject, with an integrated, first-principles approach to its five major areas: analytical dynamics, statistical mechanics, relativity, fields, and quantum theory. Examples will be drawn from many areas of physics, including Newtonian mechanics, electromagnetism, particle physics, general relativity, and quantum information. 

  \textit{\small Textbook: None}
\end{itemize} 

\section*{Montgomery College}

\subsection*{Math and Stats}

\begin{itemize}
  \item[A] \textbf{Econ 202}, \textit{Principles of Economics II}, Fall 2018
  
  Topics include supply and demand, elasticity, government controls, market failure, production, business costs, profit maximization, and market structures. 
  
  \textit{\small Textbook: None}
  
  \item[A] \textbf{Bsad 210}, \textit{Statistics for Business and Economics}, Fall 2018
  
  The meaning and role of statistics in business and economics, frequency distributions, graphical presentations, measures of central tendency and dispersion, probability, discrete and continuous probability distributions, inferences pertaining to means and proportions, regression and correlation, time series analysis, and decision theory will be discussed.
  
  \textit{\small Textbook: None}
  
\end{itemize} 


\end{document}
