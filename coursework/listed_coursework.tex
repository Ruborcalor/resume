%-------------------------------------------------------------------------------
%	SECTION TITLE
%-------------------------------------------------------------------------------
\cvsection{McGill - Comp Sci \small (CGPA 4.0)}


%-------------------------------------------------------------------------------
%	CONTENT
%-------------------------------------------------------------------------------
\begin{cventries}
  % ---------------------------------------------------------
  \cventry
  {Grade: A}
  {Comp 520 - Compiler Design}
  {Christopher Dubach}
  {Winter 2021}
  {
    \begin{cvitems} % Description(s) of tasks/responsibilities
    \item {Write a compiler in Java for a subset of C. Parser, abstract syntax tree construction, semantic analyzer, code generation, register allocation.} 
    \item {Textbook: Engineering a Compiler by Keith Cooper and Linda Torczon}
    \end{cvitems}
  }


  % ---------------------------------------------------------
  \cventry
  {Grade: A}
  {Comp 409 - Concurrent Programming}
  {Clark Verbrugge}
  {Winter 2021}
  {
    \begin{cvitems} % Description(s) of tasks/responsibilities
    \item {Atomicity, mutual exclusion, blocking, deadlock, linearizability, memory consistency, concurrent data structures, transactional memory, process algebra, dataflow.} 
    \item {Textbook: The Art of Multiprocessor Programming by Maurice Herlihy, Hir Shavit, Victor Luchangco, andMichael Spear}
    \end{cvitems}
  }


  % ---------------------------------------------------------
  \cventry
  {Grade: A}
  {Comp 362 - Honors Algorithm Design}
  {Robert Robere}
  {Winter 2021}
  {
    \begin{cvitems}
    \item {Network flows, linear programming, complexity theory, approximation algorithms, advanced topics.} 
    \item {Textbook: Algorithm Design by Kleinberg and Tardos}
    \end{cvitems}
  }


%---------------------------------------------------------
  \cventry
  {Grade: A}
    {Comp 417 - Intro to Robotics and Intelligent Systems}
    {David Meger}
    {Fall 2020}
    {
      \begin{cvitems} % Description(s) of tasks/responsibilities
      \item {Geometric motion planning, feedback control, perception, mapping, and state estimation.} 
      \item {Textbook: Introduction to Robotics, Mechanics and Control, 3rd edition, by Craig.}
      \end{cvitems}
    }

%---------------------------------------------------------
  \cventry
    {Grade: A}
    {Comp 321 - Programming Challenges} % Organization
    {David Becerra} % Location
    {Fall 2020} % Date(s)
    {
      \begin{cvitems} % Description(s) of tasks/responsibilities
      \item {Test algorithm design and programming skills on tricky problems and puzzles.}
      \item {Textbook: Programming Challenges}
      \end{cvitems}
    }

%---------------------------------------------------------
  \cventry
    {Grade: A}
    {Comp 310 - Operating Systems} % Organization
    {Muthucumaru Maheswaran} % Location
    {Fall 2020} % Date(s)
    {
      \begin{cvitems} % Description(s) of tasks/responsibilities
      \item {Theoretical and practical concepts behind modern operating systems. Processes, inter-process communication, scheduling, memory management, virtual memory, storage management, network management, and security.}
      \item {Textbook: Silberchatz and Galvin, Operating System Concepts, 10th Edition, Wiley, 2018}
      \end{cvitems}
    }

%---------------------------------------------------------
  \cventry
    {Grade: A (99th percentile)}
    {Comp 302 - Programming Languages and Paradigms} % Organization
    {Brigitte Pientka } % Location
    {Fall 2020} % Date(s)
    {
      \begin{cvitems} % Description(s) of tasks/responsibilities
      \item {Functional programming and types. Ocaml.}
      \item {Textbook: B. C. Pierce: "Types and Programming Languages". MIT Press, 2002.}
      \end{cvitems}
    }

%---------------------------------------------------------
  \cventry
    {Grade: A}
    {Comp 598 - Automata and Computability} % Organization
    {Prakash Panangaden} % Location
    {Summer 2020} % Date(s)
    {
      \begin{cvitems} % Description(s) of tasks/responsibilities
      \item {Languages, automata, Kleene theorem, minimization, monoids, linear temporal logic, reductions, PCP, valcomps, logic and unsolvability, arithmetic hierarchy.}
      \item {Textbook: Automata and Computability by Dexter Kozen}
      \end{cvitems}
    }

%---------------------------------------------------------
  \cventry
    {Grade: A} % Job title
    {Comp 252 - Honors Algorithms and Data Structures} % Organization
    {Luc Devroye} % Location
    {Winter 2020} % Date(s)
    {
      \begin{cvitems} % Description(s) of tasks/responsibilities
      \item {Introduce student to algorithmic analysis, fundamental data structures, and problem solving paradigms.}
      \item {Textbook: T.H. Cormen, C.E.Leiserson, R.L.Rivest, and C. Stein: Introduction to Algorithms (Third Edition)}
      \end{cvitems}
    }

% ---------------------------------------------------------
    \cventry
    {Grade: A} % Job title
    {Comp 273 - Intro to Computer Systems} % Organization
    {Kaleem Siddiqi} % Location
    {Winter 2020} % Date(s)
    {
      \begin{cvitems} % Description(s) of tasks/responsibilities
      \item {Number representations, combinational logic, sequential logic, MIPS assembly language and CPU architecture, Memory, I/O, Finite State Machines}
      \item {Textbook: Computer organization and design: the hardware/software interface by David A. Patterson and John L. Hennese}
      \end{cvitems}
    }

    % ---------------------------------------------------------
    \cventry
    {Grade: A} % Job title
    {Comp 322 - Introduction to C++} % Organization
    {Chad Zammar} % Location
    {Winter 2020} % Date(s)
    {
      \begin{cvitems} % Description(s) of tasks/responsibilities
      \item {Covers the essential features of C++. Focus on pointers, memory allocation, tepmlates, classes, operator overloading, namespaces, exceptions, and the STL.}
      \item {Textbook: The C++ Programming Language by Bjarne Stroustrup}
      \end{cvitems}
    }

    % ---------------------------------------------------------
    \cventry
    {Grade: A} % Job title
    {Comp 250 - Intro to Computer Science} % Organization
    {Michael Langer and Giulia Alberino} % Location
    {Fall 2019} % Date(s)
    {
      \begin{cvitems} % Description(s) of tasks/responsibilities
      \item {Learn  basic  data  structures  for  lists  (arrays,  linked  lists,  stacks,  queues),  trees  (search trees, heaps), and graphs. Analyze algorithms in terms of the amount of computation they use. Implementations in Java.}
      \item {Textbook: None}
      \end{cvitems}
    }

    % ---------------------------------------------------------
    \cventry
    {Grade: A} % Job title
    {Comp 206 - Intro to Software Systems} % Organization
    {Joseph Vybihal} % Location
    {Fall 2019} % Date(s)
    {
      \begin{cvitems} % Description(s) of tasks/responsibilities
      \item {This course focuses on System Application Development, which relates to the integration of differing software, programming languages and environments into a single application. It provides a comprehensive introduction to and overview of the C programming language and how to use it with the UNIX environment to build software. In this light the course also teaches programming in Bash, interfacing with the operating system and interfacing with networking}
      \item {Textbook: Software Systems ed 3; Vybihal \& Azar; Kendall/Hunt}
      \end{cvitems}
    }

\end{cventries}

\cvsection{McGill - Math and Stats \small (CGPA 3.96)}

\begin{cventries}


  % ---------------------------------------------------------
  \cventry
  {Grade: A} % Job title
  {Math 357 - Honors Statistics} % Organization
  {Johanna G. Neslehova} % Location
  {Winter 2021} % Date(s)
  {
    \begin{cvitems} % Description(s) of tasks/responsibilities
    \item {Random samples, point estimation, methods of evaluating estimators, sufficiency, completeness, best unbiased estimators, hypothesis testing, confidence intervals, asymptotic sampling distributions, regression analysis}
    \item {Textbook: An Introduction to Probability and Statisticsby Rohatgi and Saleh, Wiley, 2015}
    \end{cvitems}
  }


    % ---------------------------------------------------------
    \cventry
    {Grade: A} % Job title
    {Math 356 - Honors Probability} % Organization
    {Linan Chen} % Location
    {Fall 2020} % Date(s)
    {
      \begin{cvitems} % Description(s) of tasks/responsibilities
      \item {Probability, discrete and continuous random variables, probability distributions, variance, moments and generating functions, multiple random variables, independence, correlation, condition, law of large numbers.}
      \item {Textbook: An Introduction to Probability and Statisticsby Rohatgi and Saleh, Wiley, 2015}
      \end{cvitems}
    }

    % ---------------------------------------------------------
    \cventry
    {Grade: A} % Job title
    {Math 350 - Honors Discrete Math} % Organization
    {Sergey Norin} % Location
    {Fall 2020} % Date(s)
    {
      \begin{cvitems} % Description(s) of tasks/responsibilities
      \item {Fundamental concepts in graph theory: trees, matchings, connectivity, graph coloring, planar graphs.}
      \item {Textbook: Introduction to Graph Theory by D. West.}
      \end{cvitems}
    }


    % ---------------------------------------------------------
    \cventry
    {Grade: A-} % Job title
    {Math 255 - Honors Analysis 2} % Organization
    {Pengfei Guan} % Location
    {Winter 2020} % Date(s)
    {
      \begin{cvitems} % Description(s) of tasks/responsibilities
      \item {Point set topology in metric space. Sequences, convergence, and continuity in general metric space. Normed vector spaces. Riemann-stieltjes integral. Infinite series. Uniform convergence of functions, Arzela-Ascoli Compactness Theorem, Stone-Weierstrass theorem.}
      \item {Textbook: Introduction to Real Analysis, Bartle and Sherbert.}
      \end{cvitems}
    }

    % ---------------------------------------------------------
    \cventry
    {Grade: A} % Job title
    {Math 251 - Honors Algebra 2} % Organization
    {Henri Darmon} % Location
    {Winter 2020} % Date(s)
    {
      \begin{cvitems} % Description(s) of tasks/responsibilities
      \item {Focuses on linear algebra. Linear maps and matrix representations. Determinants. Canonical forms. Duality. Bilinear and quadratic forms. Real and complex inner product spaces. Diagonalization of self-adjoint operators.}
      \item {Textbook: Linear Algebra and Geometry, Kostrikin and Manin}
      \end{cvitems}
    }


    % ---------------------------------------------------------
    \cventry
    {Grade: A} % Job title
    {Math 254 - Honors Analysis 1} % Organization
    {Axel Hundemer} % Location
    {Fall 2019} % Date(s)
    {
      \begin{cvitems} % Description(s) of tasks/responsibilities
      \item {Logic, sets, functions, and other preliminaries. The Real Numbers. Sequences. Elementary Point-Set Topology. Limits and Continuity. Differentiation}
      \item {Textbook: Introduction to Real Analysis by R. Bartle and D. Sherbert, 4th edition, Wiley}
      \end{cvitems}
    }


    % ---------------------------------------------------------
    \cventry
    {Grade: A} % Job title
    {Math 235 - Algebra 1} % Organization
    {Daniel Wise } % Location
    {Fall 2019} % Date(s)
    {
      \begin{cvitems} % Description(s) of tasks/responsibilities
      \item {Sets, functions and relations. Groups, subgroups and cosets; group actions on sets. Methods of proof. Complex numbers. Divisibility theory for integers and modular arithmetic. Divisibility theory for polynomials. Rings, ideals and quotient rings. Fields and construction of fields from polynomial rings. }
      \item {Textbook: Abstract Algebra: Theory and Applications (2017 edition) by Tom Judson.}
      \end{cvitems}
    }


    % ---------------------------------------------------------
    \cventry
    {Grade: A} % Job title
    {Math 222 - Calculus 3} % Organization
    {Jerome Fortier} % Location
    {Summer 2020} % Date(s)
    {
      \begin{cvitems} % Description(s) of tasks/responsibilities
      \item {Taylor series, Taylor’s theorem in one and several variables. Review of vector geometry. Partialdifferentiation, directional derivative.  Extreme of functions of 2 or 3 variables.  Parametriccurves and arc length. Polar and spherical coordinates. Multiple integrals}
      \item {Textbook: Stewart, J.,Calculus: Multivariable Calculus, 8th Edition}
      \end{cvitems}
    }


    % ---------------------------------------------------------
    \cventry
    {Grade: A} % Job title
    {Math 133 - Linear Algebra} % Organization
    {Rosalie Bélanger-Rioux} % Location
    {Fall 2019} % Date(s)
    {
      \begin{cvitems} % Description(s) of tasks/responsibilities
      \item {Linear transformations, matrices and vectors, linearindependence, subspaces and bases. Determinants, eigenvalues and eigenvectors, diagonalization. Study of linear systems of equations, their solutions, and the underlying structure of these problems.}
      \item {Textbook: W. K. Nicholson, Linear Algebra with Applications, Open Edition. Lyryx Learning Inc. Base Textbook, Version 2019, Revision A.}
      \end{cvitems}
    }

%---------------------------------------------------------
\end{cventries}

\cvsection{McGill - Misc}

\begin{cventries}
    % ---------------------------------------------------------
    \cventry
    {Grade: A} % Job title
    {Chem 120 - General Chemistry 2} % Organization
    {Mitchell Huot, Paul Wiseman} % Location
    {Winter 2020} % Date(s)
    {
      \begin{cvitems} % Description(s) of tasks/responsibilities
      \item {Chemistry 120 aims to provide you with an introduction to the quantitative aspects of fundamental chemical principles, such as gas laws, thermodynamics, kinetics, solubility, equilibrium, and acids/base.}
      \item {Textbook: The Molecular Nature of Matter and Change (2nd Canadian Edition) by Silberberg, Amateis, Lavieri, and Venkateswaran}
      \end{cvitems}
    }

\end{cventries}

\cvsection{Harvard - Math and Stats}

\begin{cventries}
  % ---------------------------------------------------------
  \cventry
  {Grade: A} % Job title
  {Math 21b - Linear Algebra and Differential Equations} % Organization
  {Janet Chen} % Location
  {Winter 2019} % Date(s)
  {
    \begin{cvitems} % Description(s) of tasks/responsibilities
    \item {Matrices and related topics such as linear transformations and linear spaces, determinants, eigenvalues, and eigenvectors. Applications include dynamical systems including nonlinear systems, data fitting, ordinary and partial differential equations, and Fourier series.}
    \item {Textbook: Linear Algebra with Applications by Otto Bretscher}
    \end{cvitems}
  }

  % ---------------------------------------------------------
  \cventry
  {Grade: A} % Job title
  {Math 21a - Multivariable Calculus} % Organization
  {Janet Chen} % Location
  {Fall 2019} % Date(s)
  {
    \begin{cvitems} % Description(s) of tasks/responsibilities
    \item {Extending single variable calculus to higher dimensions. Develops methods for solving optimization problems with and without constraints.}
    \item {Textbook: Multivariable Calculus: Concepts and Contexts by James Stewart}
    \end{cvitems}
  }


\end{cventries}

\cvsection{Harvard - Physics}

\begin{cventries}
  % ---------------------------------------------------------
  \cventry
  {Grade: A} % Job title
  {Physics 101 - Foundations of Theroetical Physics} % Organization
  {Jacob Barandes} % Location
  {Spring 2019} % Date(s)
  {
    \begin{cvitems} % Description(s) of tasks/responsibilities
    \item {A comprehensive, fast-paced introduction to the conceptual and mathematical foundations of modern theoretical physics that starts from the very beginning of the subject, with an integrated, first-principles approach to its five major areas: analytical dynamics, statistical mechanics, relativity, fields, and quantum theory. Examples will be drawn from many areas of physics, including Newtonian mechanics, electromagnetism, particle physics, general relativity, and quantum information.}
    \item {Textbook: Lecture Notes}
    \end{cvitems}
  }

\end{cventries}

\cvsection{Montgomery College - Math and Stats}

\begin{cventries}
  % ---------------------------------------------------------
  \cventry
  {Grade: A} % Job title
  {Econ 202 - Principles of Economics II} % Organization
  {} % Location
  {Fall 2018} % Date(s)
  {
    \begin{cvitems} % Description(s) of tasks/responsibilities
    \item {Topics include supply and demand, elasticity, government controls, market failure, production, business costs, profit maximization, and market structures. }
    \item {Textbook: Lecture Notes}
    \end{cvitems}
  }

  % ---------------------------------------------------------
  \cventry
  {Grade: A} % Job title
  {Bsad 202 - Statistics for Business and Economics} % Organization
  {} % Location
  {Fall 2018} % Date(s)
  {
    \begin{cvitems} % Description(s) of tasks/responsibilities
    \item {The meaning and role of statistics in business and economics, frequency distributions, graphical presentations, measures of central tendency and dispersion, probability, discrete and continuous probability distributions, inferences pertaining to means and proportions, regression and correlation, time series analysis, and decision theory will be discussed.}
    \item {Textbook: Lecture Notes}
    \end{cvitems}
  }

\end{cventries}
